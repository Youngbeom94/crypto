% !TEX TS-program = pdflatex
% !TEX encoding = UTF-8 Unicode

% This is a simple template for a LaTeX document using the "article" class.
% See "book", "report", "letter" for other types of document.

\documentclass[11pt]{article} % use larger type; default would be 10pt

\usepackage[utf8]{inputenc} % set input encoding (not needed with XeLaTeX)

%%% Examples of Article customizations
% These packages are optional, depending whether you want the features they provide.
% See the LaTeX Companion or other references for full information.

%%% PAGE DIMENSIONS
\usepackage{geometry} % to change the page dimensions
\geometry{a4paper} % or letterpaper (US) or a5paper or....
% \geometry{margin=2in} % for example, change the margins to 2 inches all round
% \geometry{landscape} % set up the page for landscape
%   read geometry.pdf for detailed page layout information

\usepackage{graphicx} % support the \includegraphics command and options

% \usepackage[parfill]{parskip} % Activate to begin paragraphs with an empty line rather than an indent

%%% PACKAGES
\usepackage{booktabs} % for much better looking tables
\usepackage{array} % for better arrays (eg matrices) in maths
\usepackage{paralist} % very flexible & customisable lists (eg. enumerate/itemize, etc.)
\usepackage{verbatim} % adds environment for commenting out blocks of text & for better verbatim
\usepackage{subfig} % make it possible to include more than one captioned figure/table in a single float
% These packages are all incorporated in the memoir class to one degree or another...

%%% HEADERS & FOOTERS
\usepackage{fancyhdr} % This should be set AFTER setting up the page geometry
\pagestyle{fancy} % options: empty , plain , fancy
\renewcommand{\headrulewidth}{0pt} % customise the layout...
\lhead{}\chead{}\rhead{}
\lfoot{}\cfoot{\thepage}\rfoot{}

%%% SECTION TITLE APPEARANCE
\usepackage{sectsty}
\allsectionsfont{\sffamily\mdseries\upshape} % (See the fntguide.pdf for font help)
% (This matches ConTeXt defaults)

%%% ToC (table of contents) APPEARANCE
\usepackage[nottoc,notlof,notlot]{tocbibind} % Put the bibliography in the ToC
\usepackage[titles,subfigure]{tocloft} % Alter the style of the Table of Contents
\renewcommand{\cftsecfont}{\rmfamily\mdseries\upshape}
\renewcommand{\cftsecpagefont}{\rmfamily\mdseries\upshape} % No bold!

%%% END Article customizations

%%% The "real" document content comes below...

\title{Brief Article}
\author{The Author}
%\date{} % Activate to display a given date or no date (if empty),
         % otherwise the current date is printed 

\begin{document}
\maketitle

\section{Cipher Constructions}

The design of a cipher can be split into (at least) two parts: The cipher construction, and the core permutation. The two archetypal examples of cipher construction are the Feistel Network and Substiution-Permutation Network.

In the former, the state is divided into two sections that are typically of equal size; A key is added to some side(s), then a core "randomizing" function is applied to one half to produce a key stream; This key stream is then combined with the other half, in order to encipher it. We can see then, that the core "randomizing function" is interchangeable within the Feistel Network construction.

In a Substitution-Permutation network, the state is typically subdivided into many smaller sections. The algorithm typically consists of addition of a key layer, followed by at least one application of an invertible "randomizing" function applied to the entire state; This is repeated until a sufficient "security margin" is obtained. This construction can be modeled as the composition of two functions: The key addition function, and the application of the core permutation on the state. As in the Feistel Network, the core permutation is interchangeable within the Substitution-Permutation network construction.

We can see then, that there are at least two, effectively separate aspects to cipher design: The cipher construction, and the core permutation. This work focuses on the former. While the core permutation is critical for resistance to generic attacks such as linear and differential cryptanalysis, there also exist generic attacks against the cipher constructions. Additionally, while the design of the permutation can be hugely influential on the performance of the algorithm, so too can the cipher construction.

\section{Construction Design Considerations}

First, the target platform needs to be established. What is ideal for constrained hardware is not necessarily ideal for a modern CPU. In addition to the target platform(s), the performance goals of the algorithm should be outlined: Is latency important, or is throughput for large messages important? Is it important to use as little (or much) RAM? 

Less obvious concerns that are still very important include: Is resistance to side channel leakage minimized? Does it facilitate or impede attempts to implement counter-measures?



"The simple" - RC4 like (shuffling huge array of words)

"The small" - 8-bit friendly/constrained device friendly.

"The fast" - salsa like (SIMD with big registers)

\subsection{A subsection}

More text.

\end{document}

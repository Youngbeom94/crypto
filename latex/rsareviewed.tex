% !TEX TS-program = pdflatex
% !TEX encoding = UTF-8 Unicode

% This is a simple template for a LaTeX document using the "article" class.
% See "book", "report", "letter" for other types of document.

\documentclass[11pt]{article} % use larger type; default would be 10pt

\usepackage[utf8]{inputenc} % set input encoding (not needed with XeLaTeX)

%%% Examples of Article customizations
% These packages are optional, depending whether you want the features they provide.
% See the LaTeX Companion or other references for full information.

%%% PAGE DIMENSIONS
\usepackage{geometry} % to change the page dimensions
\geometry{a4paper} % or letterpaper (US) or a5paper or....
% \geometry{margin=2in} % for example, change the margins to 2 inches all round
% \geometry{landscape} % set up the page for landscape
%   read geometry.pdf for detailed page layout information

\usepackage{graphicx} % support the \includegraphics command and options

% \usepackage[parfill]{parskip} % Activate to begin paragraphs with an empty line rather than an indent

%%% PACKAGES
\usepackage{booktabs} % for much better looking tables
\usepackage{array} % for better arrays (eg matrices) in maths
\usepackage{paralist} % very flexible & customisable lists (eg. enumerate/itemize, etc.)
\usepackage{verbatim} % adds environment for commenting out blocks of text & for better verbatim
\usepackage{subfig} % make it possible to include more than one captioned figure/table in a single float
% These packages are all incorporated in the memoir class to one degree or another...

%%% HEADERS & FOOTERS
\usepackage{fancyhdr} % This should be set AFTER setting up the page geometry
\pagestyle{fancy} % options: empty , plain , fancy
\renewcommand{\headrulewidth}{0pt} % customise the layout...
\lhead{}\chead{}\rhead{}
\lfoot{}\cfoot{\thepage}\rfoot{}

%%% SECTION TITLE APPEARANCE
\usepackage{sectsty}
\allsectionsfont{\sffamily\mdseries\upshape} % (See the fntguide.pdf for font help)
% (This matches ConTeXt defaults)

%%% ToC (table of contents) APPEARANCE
\usepackage[nottoc,notlof,notlot]{tocbibind} % Put the bibliography in the ToC
\usepackage[titles,subfigure]{tocloft} % Alter the style of the Table of Contents
\renewcommand{\cftsecfont}{\rmfamily\mdseries\upshape}
\renewcommand{\cftsecpagefont}{\rmfamily\mdseries\upshape} % No bold!

%%% END Article customizations

%%% The "real" document content comes below...

\title{Alternative Perspective of the RSA Cipher}
\author{Ella Rose}
%\date{} % Activate to display a given date or no date (if empty),
         % otherwise the current date is printed 

\begin{document}
\maketitle

\begin{abstract}
 We provide an alternative description of the RSA cipher by breaking it up into a few discrete, separate pieces. We hope that by understanding RSA as part of a more generalized cipher framework, we  might come up with alternative instantiations of the cipher that utilize different hardness assumptions. We do not suggest such a design here.

\section{Introduction to RSA}
This section will outline the RSA algorithm - for the history of the design, we encourage the reader to look elsewhere. \todo{RSA history is where?}

The RSA cryptosystem utilizes three core functions:

 The key generation algorithm, which generates a private key and corresponding public key
 The encryption algorithm, which replaces the supplied number with a psuedorandom number
 The decryption algorithm, which recovers the number that was supplied to the encryption algorithm, when given the output of the encryption algorithm and the private key.

The key generation algorithm is typically described by something like the following:

Select two large, random prime numbers of approximately equal size $p, q \gets bigPrime(size) bigPrime(size)$
Calculate the totient via $totient \gets (p -1)(q - 1)$
Set e to the usual/safe value (i.e. 65537); Select the decryption exponent via $modularInverse(e, totient)
Multiply p and q to produce the modulus n $n \gets pq$
Output n, e as the public key; Output d as the public key;

While the encryption/decryption algorithms are simplier and are described by something like:

Produce a ciphertext of message by computing $message ^ e \mod n$
Produce a plaintext from ciphertext by computing $ciphertext ^ d mod n$

While this is enough information to produce a working implementation of the "textbook" RSA algorithm, it is does not really fit well into a generic cipher framework. 

\section{An Alternative Introduction to RSA}
First, we shall describe a simple secret key cipher that is homomorphic with respect to multiplication of ciphertexts:

Select a random prime number of sufficient size $p \gets bigPrime(size)$
Select an encryption exponent of appropriate size (>2 and <size of p)
Calculate the decryption exponent by computing $d \gets modularInverse(e, p - 1)$
Encrypt a number by computing $m^e mod n$ and decrypt by computing $c^d mod n$

The only difference so far is that we are using a single prime number as the modulus. Despite the fact that encryption and decryption are asymmetric operations, this cipher is not secure as a public key cipher, as anyone who knows $p$ and $e$ may compute $d$.

RSA typically computes the decryption exponent by computing $modularInverse(e, totient(n))$, but it works equally well to use $modularInverse(e, totient(p))$ or $modularInverse(e, totient(q))$ - as long as the message is smaller then p or q, respectively.

So the modulus $n \gets np$ that is typically released as part of the public key is effectively an encrypted modulus.

As long as we encrypt messages of size smaller then p (the basis), then we do not need to know the totient of n - we only need to know the totient of p (or at least, one factor of n).

# the purpose of q is to encrypt p so that others cannot obtain d
# mk is a secure cipher if m and k are large primes and no "fast" classical algorithm for factoring exists


\end{document}

\documentclass[preprint]{iacrtrans}
\usepackage[utf8]{inputenc}

% Select what to do with todonotes: 
% \usepackage[disable]{todonotes} % notes not showed
\usepackage[draft,color=orange!20!white,linecolor=orange,textwidth=3cm,colorinlistoftodos]{todonotes}   % notes showed
\setlength{\marginparwidth}{3cm}
\usepackage{graphicx}
\usepackage{soul}
\graphicspath{{images/}} % end dirs with `/'
% \usepackage{longtable}
\usepackage{tikz}
\usetikzlibrary{arrows}
\usetikzlibrary{arrows.meta}
\usetikzlibrary{positioning}
\usetikzlibrary{calc}
\usetikzlibrary{backgrounds}
\usetikzlibrary{arrows}
\usetikzlibrary{crypto.symbols}
\tikzset{shadows=no}        % Option: add shadows to XOR, ADD, etc.

\author{Ella Rose\inst{1}}
\institute{Paso Robles, CA\email{python_pride@protonmail.com}}
\title[Secret-Key Based Public-Key Encryption]{Secret-Key Based Public-Key Encryption}

\begin{document}

\maketitle

% use optional argument because the \LaTeX command breaks the PDF keywords
\keywords[Homomorphic Encryption, Symmetric Encryption, Asymmetric Encryption]{Homomorphic Encryption, Symmetric Encryption, Asymmetric Encryption}

\begin{abstract}
 We present a perspective which demonstrates not only that secret-key based public key encryption is possible, but that well known primitives already fit into this framework. We describe a variety of such constructions, which are of varying complexity and security.
\end{abstract}

\todototoc
\listoftodos

\section{Introduction}
 Ciphers that support homomorphic operations on ciphertexts are very attractive because they tend to make the creation of public key encryption schemes relatively straightforward and simple. This paper attempts to illustrate and bring to light this fact. We will begin by covering "classic" cryptosystems such as Merkle's original knapsack cryptosystem, as well as modular exponentiation based cryptosystems such as DH and RSA.

\section{Definitions}
We use $+$ to denote addition, $ab$ to denote multiplication, $a / b$ to denote division, and $\mod p$ to denote the modulo operation. Exponentiation is denoted as $a \pow b$.

\section{Merkle's Knapsack Cryptosystem}
\subsection{Basic Concept}
This original cryptosystem is actually slightly more complex then some of the others that will be described. This is largely due to the fact that decryption involves actually solving a subset sum problem.

The secret key cipher in this cryptosystem is modular multiplication. The plaintext numbers, as well as the modulus, are kept secret, in addition to the key.

To create a public key, the secret key cipher is used to encrypt a set of numbers. The numbers are chosen such that utilizing the set to perform a subset sum problem will result in an easy problem to solve, allowing recovery of the summed elements. After the numbers have been encrypted, the set of numbers will be randomized and using them in a subset sum will result in greater difficulty inverting the sum.

The public key encryption process involves summing elements of the public key according to the bits of the message to be sent. Decrypting a ciphertext produced this way is equivalent to solving a subset sum problem.

The secret key is used to turn the summed elements back into the "easy version" of the numbers, so that the subset sum problem can then be solved.

\subsection{Key Generation}
\subsubsection{Encryption}
\subsubsection{Decryption}

\subsection{Randomization}
Any homomorphic cipher requires randomization in order to be secure. The presented design is very simple to randomize; Continuing with the previous example, we would simply generate an additional $k_3$ and a random $r_0$ and include that in the sum as well:

\section{Conclusion}

\end{document}


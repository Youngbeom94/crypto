\documentclass[preprint]{iacrtrans}
\usepackage[utf8]{inputenc}

% Select what to do with todonotes: 
% \usepackage[disable]{todonotes} % notes not showed
\usepackage[draft,color=orange!20!white,linecolor=orange,textwidth=3cm,colorinlistoftodos]{todonotes}   % notes showed
\setlength{\marginparwidth}{3cm}
\usepackage{graphicx}
\usepackage{soul}
\graphicspath{{images/}} % end dirs with `/'
% \usepackage{longtable}
\usepackage{tikz}
\usetikzlibrary{arrows}
\usetikzlibrary{arrows.meta}
\usetikzlibrary{positioning}
\usetikzlibrary{calc}
\usetikzlibrary{backgrounds}
\usetikzlibrary{arrows}
\usetikzlibrary{crypto.symbols}
\tikzset{shadows=no}        % Option: add shadows to XOR, ADD, etc.

\author{Ella Rose\inst{1}}
\institute{Paso Robles, CA\email{python_pride@protonmail.com}}
\title[Secret-Key Based Public-Key Encryption]{Secret-Key Based Public-Key Encryption}

\begin{document}

\maketitle

% use optional argument because the \LaTeX command breaks the PDF keywords
\keywords[Homomorphic Encryption, Symmetric Encryption, Asymmetric Encryption]{Homomorphic Encryption, Symmetric Encryption, Asymmetric Encryption}

\begin{abstract}
 We present a perspective which demonstrates not only that secret-key based public key encryption is possible, but that well known primitives already fit into this framework. We describe a variety of such constructions, which are of varying complexity and security.
\end{abstract}

\todototoc
\listoftodos

\section{Introduction}
 Ciphers that support homomorphic operations on ciphertexts are very attractive because they tend to make the creation of public key encryption schemes relatively straightforward and simple. This paper attempts to illustrate and bring to light this fact. We will begin by covering "classic" cryptosystems such as Merkle's original knapsack cryptosystem, as well as modular exponentiation based cryptosystems such as DH and RSA.

\section{Definitions}
We use $+$ to denote addition, $ab$ to denote multiplication, $a / b$ to denote division, and $\mod p$ to denote the modulo operation. Exponentiation is denoted as $a ^ b$.

\section{General Properties of Homomorphic Ciphers}
\subsection{Basic Definition of Homomorphic Cipher}
A ciphertext is homomorphic if it supports $D(E(m1) * (m2)) == m1 ** m2$ where $x * y$ and $x ** y$ are some operation(s) such as addition or multiplication. 

A partially homomorphic cipher is usually defined as supporting an unlimited quantity of a single type of operation on ciphertexts, such as $D(E(m1) + E(m2)) == m1 + m2$.

We will note that such a property implies the ability to perform another operation with plaintext values as the second operand, as in $D(E(m1) * m2)$ == m1 * m2. Put in other words, suppose that addition can be performed on ciphertexts; Then the sum of a ciphertext with itself would be equivalent to multiplication of the ciphertext by 2. Thus the ability to add ciphertexts to each other necessarily implies the ability to multiply a ciphertext by a plaintext amount.

\subsection{Access to Encryption Oracle}
Interestingly, with homomorphic ciphers, access to a ciphertext can be effectively equivalent to access to an encryption oracle, in that the one ciphertext can be manipulated to create new ciphertexts, the plaintext values for which will have a known relation to the previous. So in this sense, homomorphic ciphers must be secure against a type of adaptive chosen plaintext attack even if only a single plaintext is released.

\subsection{Randomization}
All homomorphic ciphers requires randomization in order to be secure. 

\section{Merkle's Knapsack Cryptosystem}
\subsection{Basic Concept}
This original cryptosystem is actually slightly more complex then some of the others that will be described. This is partly due to the fact that decryption involves actually solving a subset sum problem.

The secret key cipher in this cryptosystem is modular multiplication. The plaintext numbers, as well as the modulus, are kept secret, in addition to the key.

To create a public key, the secret key cipher is used to encrypt a set of plaintext numbers. The numbers are chosen such that utilizing the set to perform a subset sum problem will result in an easy problem to solve\footnote{\label{superincreasing sequence}A superincreasing sequence}, allowing recovery of the summed elements. After the numbers have been encrypted, the set of numbers will appear randomized and using them in a subset sum will result in greater difficulty inverting the sum.

The public key encryption process involves summing elements of the public key according to the bits of the message to be sent. Decrypting a ciphertext produced this way is equivalent to solving a subset sum problem.

The secret key is used to convert the summed elements back into the "easy version" of the numbers, so that the subset sum problem can then be solved. The secret key acts as a sort of toggle that changes the numbers from one representation to another. 

The plaintext numbers and modulus must be kept secret, because otherwise modular multiplication does not make a strong enough cipher. A more general and flexible approach to keeping than plaintext a secret is randomization. Indeed, all homomorphic ciphers require randomization in order to be secure against chosen plaintext attack. 

\subsection{Key Generation}
\subsubsection{Encryption}
\subsubsection{Decryption}

\section{Modular Exponentiation}
Exponentiation modulo a prime is, at it's simplest, equivalent to a block cipher. In this cipher, the base is the message to be encrypted, while the encryption key is the exponent, and the modulus selects the block/key size. Ciphertexts are homomorphic with respect to multiplication, and encryption is commutative.

\subsection{Diffie-Hellman}
With this perspective, we can see that Diffie-Hellman key exchange can be described as follows: Each party selects a random encryption key and encrypts the same message, generating independently two ciphertexts $c_1$ and $c_2$. The parties exchange these ciphertexts, and each encrypts the others ciphertext using their own key.

The result of this at either end point yields a ciphertext that has been encrypted by both keys, albeit in a different order. Because encryption with this cipher is commutative, the fact that the encryption operations are applied in alternative sequencing does not matter.

Note that the homomorphic property of ciphertexts was not really utilized here - only the fact that encryption is a commutative block cipher seems to be required. Additionally, since this is only a key agreement scheme, the parties are not interested in encrypting any specific messages, or inverting any encrypted messages to the plaintexts; Both parties simply need to both arrive at the same, otherwise random value. The requirements for a key agreement algorithm appear to be simpler or easier to meet than the requirements for a public key encryption algorithm.

\subsection{RSA}
Like Diffie-Hellman, RSA utilizes modular exponentiation, but utilizes the same very simple expression quite differently. The basic modular exponentiation cipher uses the base as the message, the exponent as the encryption key, and the modulus as a block size/key size selector. There is a natural asymmetry between encryption and decryption in this cipher, in that decryption requires the modular inverse of the encryption exponent. 

The problem, however, is that (in the basic cipher) anyone who knows the encryption exponent and the prime modulus can compute the decryption exponent. In order to create an asymmetric algorithm, it must be difficult to compute the decryption exponent. The solution is simple: Encrypt the modulus.

Multiplication of two integers is, in the worst case scenario, a "hard" problem. 

\section{Lattice Based}
\subsection{Encryptions of 0}
\subsection{Encryptions of range(N)}
\subsection{Encryptions of 2 ** i for i in range(n)}
\subsection{Encrypted array/superposition (multi-mod)}
\subsection{Encrypted 1 And 0s} % p1m + p2r + p3r % p1 = encrypt(1); p2 = encrypt(0); p3 = encrypt(0);
\subsection{Key Exchange} % p1q1 + p2q2 (where the shared secret is q1 + q2)
\subsection{Key exchange 2} % p1q1 + p2q2 + e (D(E(m) + m) == m + m is required)


\section{Conclusion}

\end{document}


\documentclass[preprint]{iacrtrans}
\usepackage[utf8]{inputenc}

% Select what to do with todonotes: 
% \usepackage[disable]{todonotes} % notes not showed
\usepackage[draft,color=orange!20!white,linecolor=orange,textwidth=3cm,colorinlistoftodos]{todonotes}   % notes showed
\setlength{\marginparwidth}{3cm}
\usepackage{graphicx}
\usepackage{soul}
\graphicspath{{images/}} % end dirs with `/'
% \usepackage{longtable}
\usepackage{tikz}
\usetikzlibrary{arrows}
\usetikzlibrary{arrows.meta}
\usetikzlibrary{positioning}
\usetikzlibrary{calc}
\usetikzlibrary{backgrounds}
\usetikzlibrary{arrows}
\usetikzlibrary{crypto.symbols}
\tikzset{shadows=no}        % Option: add shadows to XOR, ADD, etc.

\author{Ella Rose\inst{1}}
\institute{Paso Robles, CA\email{python_pride@protonmail.com}}
\title[\texttt 4-16-17]{\texttt 4-16-17}

\begin{document}

\maketitle

% use optional argument because the \LaTeX command breaks the PDF keywords
\keywords[Permutation, Cipher Construction, AEAD]{Permutation, Cipher Construction, AEAD}\todo{Keywords?}

\begin{abstract}
 We define a permutation based construction for providing single pass authenticated encryption of data. Our specific definition recommends a 512-bit permutation and a 256-bit key to encrypt a 256-bit block of data. Our construction is very simple and easy to implement; It should be significantly less code then a block cipher with a mode of operation and HMAC. It is generic, and can be instantiated with any psuedorandom permutation.
\end{abstract}

\todototoc
\listoftodos

\section{Introduction}
 Psuedorandom permutations are a core building block of modern cryptography. They are utilized to create block ciphers in Substitution-Permutation networks, as well as to create hash functions via the Sponge construction. By separating the design of cipher and hash constructions from the design of permutations, we can simplify the responsibilities of each component. The permutation holds the responsibility to be designed to resist all known generic attack frameworks, such as linear and differential cryptanalysis. 

In this work we present a construction that utilizes a psuedorandom permutation to create a simple streaming authenticated encryption system.

\section{Definitions}
\todo{Define the notation used in this paper}\\
\todo{Define how the state is laid out and indexed/addressed}\\
We use $\oplus$ to denote XOR and $||$ to denote concatenation.

\section{Algorithm}
\begin{Algorithmic}
\Function{$encrypt$}{$data, key, iv$}
	\State $k \gets key$
	\For{

\section{Design\ Rationale}
\todo{Explain the reasoning for each step of the algorithm}\\
\todo{Explain how constants are generated and what purpose they serve}\\
\todo{Describe why the linear and non-linear layers were designed/chosen}\\
\todo{Justify numerical design decisions, such as word transpositions indices and rotation amounts}\\

\section{Conclusion}
 We define an N-bit permutation that is oriented towards PLATFORM TYPE. \todo{Fill in values for block size and platform type} 
\todo{Concisely summarize the rest of the paper in a paragraph or two}

\end{document}


\documentclass[preprint]{iacrtrans}
\usepackage[utf8]{inputenc}

% Select what to do with todonotes: 
% \usepackage[disable]{todonotes} % notes not showed
\usepackage[draft,color=orange!20!white,linecolor=orange,textwidth=3cm,colorinlistoftodos]{todonotes}   % notes showed
\setlength{\marginparwidth}{3cm}
\usepackage{graphicx}
\usepackage{soul}
\graphicspath{{images/}} % end dirs with `/'
% \usepackage{longtable}
\usepackage{tikz}
\usetikzlibrary{arrows}
\usetikzlibrary{arrows.meta}
\usetikzlibrary{positioning}
\usetikzlibrary{calc}
\usetikzlibrary{backgrounds}
\usetikzlibrary{arrows}
\usetikzlibrary{crypto.symbols}
\tikzset{shadows=no}        % Option: add shadows to XOR, ADD, etc.

\author{Ella Rose}
\institute{Paso Robles, CA \email{python_pride@protonmail.com}}
\title[LWE Trapdoor One-Way Function]{LWE Trapdoor One-Way Function}

\begin{document}

\maketitle

% use optional argument because the \LaTeX command breaks the PDF keywords
\keywords[Homomorphic Encryption, Public Key Cryptography]{Homomorphic Encryption, Public Key Cryptography}

\begin{abstract}
 We define a simple trapdoor function based on the Learning-With-Errors problem in one dimension, and present a key exchange algorithm based on the trapdoor. The trapdoor algorithm is efficient in terms of key size, key generation time, ciphertext size, and public/private key operation time: We can perform many thousands of public and private key operations per second with a naive python implementation on modest hardware. 
 \end{abstract}

\todototoc
\listoftodos


\section{Introduction}
A Trapdoor One-Way Function (TOWF) is a function that is easy for anyone to compute in one direction, but is hard to compute in reverse without some additional piece of information. A TOWF makes exchanging cryptographic keys a straightforward exercise: Suppose that Bob happens to know the trapdoor information for a TOWF: Alice can simply apply the TOWF to a randomly generated key and send the output to Bob. Bob can then use the trapdoor information to invert the output and obtain the key.  

The Learning-With-Errors (LWE) problem is a mathematical problem that is believed to be difficult to solve efficiently even with a quantum computer. 

In this paper we define a candidate TOWF based on 1-dimensional LWE. It is based on a simple observation of a congruence, $pq + e \equiv q + p_i e \mod N$, along with some subtle manipulation of the sizes of the numbers involved.

\section{Definitions}
We $+$ to denote addition, $a b$ to indicate multiplication, $a / b$ to denote division, and $\mod p$ to denote the modulo operation. We use the function $inverse(k, n)$ to indicate the computation of the modular multiplicative inverse of $k \mod n$. We use the function $random(k)$ to indicate the generation of of a random $k$-bit integer.

\section{Algorithm}
There is a publicly agreed upon modulus $N$. The same $N$ may always be used. It should be prime; It it is not secret, and otherwise only needs to be both agreed upon and large enough. The size of $N$ is determined by the rest of the parameter sizes which are discussed below; In our python implementation which targets a 256-bit security level, $N$ is slightly less then 800 bits in size. The larger $N$ is, the longer the BigNum arithmetic operations take, and the larger ciphertexts become; Consequently, we opt for the smallest $N$ possible.

\subsection{Basic Concept}
Consider the following function:

\begin{flalign*}
c := pq + e \mod N
\end{flalign*}

Given $c, p, N$, we can recover $q, e$ using the following algorithm:

\begin{flalign*}
p_i := inverse(p, N)\\
t := p_i c \equiv p_i p q + p_i e \equiv q + p_i e \mod N\\
q := t \mod p_i\\
t := t - q \\
e := t / p_i
\end{flalign*}

The step $q := t \mod p_i$ will only function correctly if $q < p_i$ and if $p_i e < N$. We leverage this latter fact to create a trapdoor. We select a suitable $p_i$ that is large enough such that $q < p_i$ and small enough such that $p_i e < N$. Then, we calculate the modular multilicative inverse $p := inverse(p_i, N)$. This number will likely be very large, closer to $N$ in size then to $p_i$. To create a public key, we then encrypt $p$ using a random $r$ to preclude the ability of others to calculate $p_i$. 

The public key provides the ability to compute the forward operation $p q + e \mod N$ because there is no size constraint on $p$ for this operation (as long as $p < N$ anyways). Supplying the inverse of the public key to the inversion algorithm will not produce correct plaintext, because the generated value will be too large. We can witness this via the attempt below:

\begin{flalign*}
q := random(q_{size})\\
e := random(e_{size})\\
c := prq + e \mod N\\
pr_i := inverse(pr, N)\\
c_i := c pr_i \mod N\\
c_i \equiv q + p_i r_i e \mod N\\
\end{flalign*}

In the final line $c_i \equiv q + p_i r_i e \mod N$, we can see that the inverse of the public key is too large if $p_i r_i e > N$. While $p_i$ is relatively small, $r_i$ is (most likely) large - Ultimately, the size of $p_i r_i$ is roughly the same as $p r$, and so this alternative perspective of $q + p_i r_i e$ offers no fewer unknown bits to guess then the original $prq + e$ did.

This is in contrast to the perspective that the holder of $p_i$ has:

\begin{flalign*}
q := random(q_{size})\\
e := random(e_{size})\\
c := prq + e \mod N\\
p_i := inverse(p, N)\\
c_i := c p_i \mod N\\
c_i \equiv qr + p_i e \mod N
\end{flalign*}

We can see the difference in the final line $c_i \equiv qr + p_i e \mod N$ from before; While the $r$ variable is still present on the left term, $p_i e < N$ and so the inversion algorithm will work as intended. The presence of the $r$ variable on the left does require that $p_i$ be larger to ensure that $qr < p_i$.

\subsection{Key Generation}
\begin{flalign*}
k_{priv} := random(priv_{size})\\
p := inverse(k_{priv}, N)\\
r := random(r_{size})\\
k_{pub} := p r \mod N\\
\end{flalign*}


$priv_{size}$ must be selected such that $e k_{priv} < N$, as well as $k_{priv} > r q$, where $q, e$ are the security parameter of the public key operation and shared secret. For 256-bit $e, r, q$ values, $priv_{size}$ is 520 bits, while $N$ is slightly less then 800 bits.

\subsection{Exchanging Keys}
To exchange a secret key: multiply the public key $p$ by a random value $q$, then add a random value $e$ to the result modulo $N$. The shared secret will be either $q$ or $e$ (or some combination of the two); We discuss the implications of the choices later. We default to using $e$ as the secret, as it is slightly less complex.

\begin{flalign*}
p := key_{pub}\\
q := random(q_{size})\\
e := random(e_{size})\\
ciphertext := p q + e \mod N\\
\end{flalign*}

\subsection{Key Recovery}
The shared secret $e$ (or $q$ or both) can be recover from the ciphertext via the following algorithm:

\begin{flalign*}
p_i := key_{priv}\\
t := ciphertext * p_i \mod N\\
q := t \mod p_i\\
t := t - q\\
e := t / pi\\
q := q / r
\end{flalign*}

\subsubsection{Processing the Secret}
The secret should be processed before use a key. Ideally, HKDF should be used; If it is not available, a simple hash function applied to the secret should be sufficient.

\subsection{Homomorphism}
In addition to the ability to add ciphertexts together, ciphertexts can be multiplied by plaintext values, and plaintext values can be added to ciphertexts. This second property follows naturally from the ability to add ciphertexts together. The third property is a result of the fact that $e$ is attached to the ciphertext by simple addition. In the following examples, $c_i$ are ciphertexts, while $1, 2$ are plaintext integers.

\begin{flalign*}
c_3 := c_1 + c_2\\
c_4 := c_3 + 1\\
c_5 := 2 c_4 
\end{flalign*}

It should be noted that multiplication will quickly increase the size of the terms in the expressions, and if the size of the terms grows too large, decryption will fail.

\subsubsection{$q$ and Anti-Malleability}
If $q$ is used as the secret instead of $e$, then the ability to add plaintext values to ciphertexts disappears. The value of $e$ will not influence $q$, and manipulations of the ciphertext less then a certain magnitude will not modify the plaintext obtained upon decryption. This is in stark contrast to the more common Non-Malleability, where modification of the ciphertext will usually modify the corresponding plaintext to a meaningless value.

\section{Security Analysis}
There are two main goals: First, given the public key, the private key must not be obtainable; Second, The output of the public key operation must not be invertible without the private key.

\subsection{Key Generation}
It would appear that obtaining the private key from the public key must be done by guessing one of $p, p_i, r$. If this were not the case, then whatever algorithm that can be used to recover any of $p, p_i, r$ from $p r \mod N$ could be used to recover plaintexts from ciphertexts of the form $m k \mod N$ for random $m, k$ where one of $m, k$ is smaller then the other (but still large enough to reduce modulo $N$). It seems unlikely that such an attack exists and is faster then guessing $p, p_i,  r$. 

This implies that there does not exist an algorithm to recover the private key from the public key, other then brute force guessing.

\subsection{Public Key Operation}
The public key operation appears to be a 1-dimensional version of the learning-with-errors problem.\\

If we consider the case where $pr = 1$, then $prq + e \equiv q + e \mod P$ for uniformly random values $q, e$. 

\subsection{Side Channel Analysis}
Assuming that BigNum operations can be implemented in a manner that does not leak timing information, it appears that a naive implementation of the algorithm is resistant to timing side channel analysis, as there are no tables or branches in the algorithm.

\section{Metrics}
Using our public domain python implementation that is configured for a 256-bit security level, we can perform 10,000 public key operations in slightly under 1 second. It requires .3 seconds to perform 10,000 private key operations, which implies that roughly 30,000 private key operations per second can be performed on very modest hardware. Our tests were performed on an Intel Celeron N2830 2.16GHz with 2 cores. 

In our implementation, most of the time for the public key operation is spent generating $q$ and $e$. In a synthetic test where $q, e$ are not generated for each invocation of the function, we can perform 100,000 public key operations per second.

Our implementation is optimized for readability. Much of the computation time is spent on bignum operations, and while an optimized C implementation may certainly be faster, it is not required to obtain an acceptable level of performance. 

\subsection{Implementation Complexity}
The implementation complexity of the scheme is very low. Our public domain python implementation of the algorithm spans barely 100 lines, including the documentation and comments.\\

Additional code is required for supporting functions such as the generation of prime numbers (only if $N$ is being generated/not already established) and unit testing. These are modest requirements that can be implemented without much complexity. \\

\section{Conclusion}
We define a conceptually simple, resource-practical Trapdoor One-Way Function and key exchange algorithm. 
\end{document}

